\documentclass[a4paper,12pt]{article}
\usepackage[utf8]{inputenc}
\usepackage{amsmath}
\usepackage{amssymb}
\usepackage{geometry}
\geometry{margin=1in}

\begin{document}

\title{Cálculo de Pérdidas de Calor en un Edificio}
\author{}
\date{}
\maketitle

\section*{Datos del problema}
\begin{itemize}
    \item Área de cerramientos: \( A = 690 \, \text{m}^2 \)
    \item Transmitancia térmica: \( U = 1.2 \, \text{W/m}^2 \cdot {}^\circ\text{C} \)
    \item Temperatura base de calefacción: \( T_b = 18 \, {}^\circ\text{C} \)
    \item Temperaturas medias:
    \begin{itemize}
        \item Noviembre: \( T = 12 \, {}^\circ\text{C} \)
        \item Diciembre: \( T = 8 \, {}^\circ\text{C} \)
        \item Enero: \( T = 6 \, {}^\circ\text{C} \)
        \item Febrero: \( T = 10 \, {}^\circ\text{C} \)
    \end{itemize}
\end{itemize}

\section*{Cálculo de los Grados-día de calefacción (GDC)}
\[
\text{GDC} = \text{Días del mes} \times (T_b - T_{\text{media}})
\]

\begin{align*}
\text{Noviembre:} & \quad 30 \times (18 - 12) = 180 \, \text{GDC} \\
\text{Diciembre:} & \quad 31 \times (18 - 8) = 310 \, \text{GDC} \\
\text{Enero:} & \quad 31 \times (18 - 6) = 372 \, \text{GDC} \\
\text{Febrero:} & \quad 28 \times (18 - 10) = 224 \, \text{GDC} \\
\text{Total:} & \quad 1,086 \, \text{GDC}
\end{align*}

\section*{Cálculo de las pérdidas de calor}
La fórmula para calcular las pérdidas de calor es:
\[
q = U \cdot A \cdot \text{GDC} \cdot 24
\]

Sustituyendo los valores:
\[
q = 1.2 \cdot 690 \cdot 1,086 \cdot 24 \, \text{Wh}
\]

Realizando los cálculos:
\begin{align*}
1.2 \cdot 690 & = 828 \\
828 \cdot 1,086 & = 899,808 \\
899,808 \cdot 24 & = 21,595,392 \, \text{Wh}
\end{align*}

Convertimos a kWh:
\[
q = \frac{21,595,392}{1,000} = 21,595 \, \text{kWh/año}
\]

\section*{Resultado final}
La pérdida de calor del edificio es:
\[
\boxed{21,595 \, \text{kWh/año}}
\]

\end{document}
